\documentclass[10pt]{article}
\usepackage[OT2,T2A]{fontenc}
\usepackage{html}


\title{ TermWare: Semantics Description  \newline
        \small{DocumentId:GradSof-TermWare-e-Sm-7.10.2002-5}
      }


\newcommand{\CmBase}{../..}
\newcommand{\APIBase}{\CmBase/jdocs/ua/gradsoft/termware}
\newcommand{\TAPIBase}{\CmBase/docs/TAPI/}


\bibliographystyle{plain} 

\begin{document}

                                               
\maketitle{}

\tableofcontents

\section{  Introduction }

 Termware is computer algebra system designed for embedding into language processing
 systems. The formal model, that underlies language semantics, is defined in the article. 
 Also TermWare syntax and internal Java API structure are briefly described here. The 
 article was meant to acquaint a reader both with object-oriented and functional 
 programming and with mathematical logic fundamentals. But if you are --- a programmer, 
 who wants to use TermWare in your own projects you can skip the reading of the article 
 except of the first paragraph.

\section{  Formal Model  }

 The usual terminal algebra is built here, where the language objects are terms (expressions 
 of $f_i(x_1\dots x_n)$ form, variable and elementary data types, the same as those of all 
 functional programming languages). And its only difference from standard 
 approach is that the ordered set constructor is the dedicated term and substitution 
 operations retain order on those terms.

        
 The rest of the part will be devoted to wearing construction:

\subsection{ Alphabet }
\begin{enumerate}
 \item Primitive types set  (note , it isn't many-sorted algebra ) 
  
            $\Sigma_{pt}= \{ CHAR, SHORT, INT, BigDecimal, BigInteger, BOOL, STRING, ATOM \}$
 \item Typed constants set  $\Sigma_c^{pt} = \{ c_i^{pt} \}$, 
 
 each constant is linked with the value on corresponding primitive type set, i.e.
   $val:\Sigma_c^{pt} -> Dom^{pt}$, where  
   \begin{itemize}
     \item $Dom^{<Primitive Type>}$ - set of primitive types, represented by appropriative constants.
     \item $Dom^{ATOM}$ - atomic interpreted values set.
      Among atoms let's mark out \verb|NIL| atom , that plays a special role.    
   \end{itemize}
 \item Terminal symbols set  $\Sigma_t = \{ t_i \}$
 \item Substitute symbols set  $\Sigma_x = \{ x_i \}$
 \item Parenthesis '(' and  ')' and comma ','
 \item Functional symbols set  $\Sigma_f = \{ f_i \}$
\end{enumerate}


\subsection{ Terms }

Let's define concrete terms set $T_c$ :
\begin{itemize}
 \item $\forall c_i \in \Sigma_c$, $c_i \in T_c$
 \item $\forall t \in \Sigma_t, \forall r \in T_c^{+} = \{T_1, \dots T_n\} $
         $t(T_1,\dots T_n) \in T_c$ 
\end{itemize}

And  substitute terms set  $T_v$
\begin{itemize}
 \item $\forall t \in T_c \{ t \in T_v \}$
 \item $\forall x \in \Sigma_x$, $x \in T_v$
 \item $\forall t \in \Sigma_t, r \in T_v^{+} = \{T_1,\dots T_n\} $
         $t(T_1,\dots T_n) \in T_v$
\end{itemize}

 Term, which belongs to $T_v / T_c$ we will call term with free variables. For each term $T$ 
the set of substitute symbols, that form the term, will call the set of free variables of the term and 
will mark as $v(T)$. 

\subsection{ Operations }
Let's define next model functions of our term system: 
 \begin{itemize}
  \item $typename0: T_v \to INT$ which map different types to different integers.
  \item $name: T_v \to STRING$
  \begin{itemize}
    \item $name(c_i)=STRING(c_i)$
    \item $name(x_i)=STRING(x_i)$
    \item $name(t_i(1_1,\dots r_n))=STRING(t_i)$
  \end{itemize}
  \item $arity: T_v \to INT$
  \begin{itemize}
    \item $arity(c_i)=0$
    \item $arity(x_i)=0$
    \item $arity(t(x_1,\dots x_n))=n$
  \end{itemize}
  \item $subterm: T_v \times \{INT\} \to T_v$
  \begin{itemize}
    \item $\forall j\in \{INT\}: subterm(c_i,j)=NIL$
    \item $\forall j\in \{INT\}: subterm(x_i,j)=NIL$
    \item $\forall t\in T_v, j\in \{INT\}: j<1 \to subterm(t,j)=NIL$
    \item $$\forall t in T_v, j\in \{INT\}: 0<j<arity(t) \to subterm(t(t_1,\dots,t_n),j)=t_j$$  if suppose that there is $t_j$ subterm on  $j$-th position .
    \item $\forall t\in T_v, j\in \{INT\}: j>=arity(t) \to subterm(t,j)=NIL$
  \end{itemize}
  \item $equal T_v\times T_v \to BOOL$ ( For $equal(x,y)$  we assign 
           shortening   $x==y$ 
  \begin{itemize}
    \item $c_i == c_j$, if $c_i$ and $c_j$ represent the same model  object.
    \item $x_i == x_j$, always.
    \item $t_1(r_1,\dots,r_n) == t_2(q_1,\dots q_k)$, if $t_1$ and  $t_2$ the same
     terminal symbol, $r==n$ \& $\forall i \in {1\dots n} r_i == q_i$
  \end{itemize}
  \item $less: T_v\times T_v \to BOOL$ 
      (For  $less(x,y)$ we assign shortening  $x<<y$)
  \begin{itemize}
    \item $\forall x !=NIL, NIL << x $
    \item $\forall c_i, c_j , typename(c_i)==typename(c_j) \to c_i < c_j \Leftrightarrow c_i << c_j  $, where  $<$ is natural model area  regulation.
    \item $\forall t_1,t_2: typename0(t_1)='atom', typename0(t_2) != 'atom' t_1 << t_2$
    \item $\forall t_1,t_2: typename0(t_1)='string', typename0(t_2) != 'atom' \land typename0(t_2) != 'string' : t_1 << t_2$
    \item $\forall t_1,t_2: typename0(t_1)='boolean', typename0(t_2) != 'atom' \land typename0(t_2) != 'string' \land typename(0,t_2)!='boolean': t_1 << t_2$
    \item $\forall t_1,t_2: typename0(t_1)='int', typename0(t_2) != 'atom' \land typename0(t_2) != 'string' \land typename0(t_2)!='boolean' \land typename0(t_2)!='int': t_1 << t_2$
    \item $\forall t_1,t_2 \in T_v: arity(t_1) < arity(t_2) \to t_1 < t_2$
    \item $\forall t_1(t_11,\dots,t_1n),t_2(t_21,\dots,t_2n) \in T_v, t_1.getName() < t2.getName() \to t1 << t2$
    \item $\forall t_1(t_11,\dots,t_1n),t_2(t_21,\dots,t_2n), T_v, t_1.getName() = t2.getName() : t_11 << t_21 \to t1 << t2$
    \item $\forall t_1(t_11,\dots,t_1n),t_2(t_21,\dots,t_2n), T_v, t_1.getName() = t2.getName() \land t_11 = t_21 : t_1(t_12 \dots t_1n) << t_2(t_22 \dots t_2n) \to t1 << t2$
    \item $\forall t_1,t_2: t_1==t_2 \Leftrightarrow \not(t_1 <<t2) \land \not(t_2<<t_1))$
  \end{itemize}
    i. e. term set is just partly regulated. Which allows to designate the ordered set constructor.
    \item  $set_n(t_1,\dots,t_n)$ is the term, that signify ordered set of  $n$ elements. 
    $$s=set_n(t_1,\dots,t_n): \forall 0i<j<n \; s.subterm(i)<<s.subterm(j)$$
  \item Now we can define  $fv(T_v)\to T_v$  - the free variables ($t$) set.
    \begin{itemize}
      \item $fv(c_i)=NIL$
      \item $fv(x_i)=set_1(x_i)$
      \item $fv(f(t_1,\dots,t_n))=set(x_i)\cup freeset(f(t_2,\dots,t_n))$
    \end{itemize}
    And $fvi(x)$ is term indexes set .
 \item Each subterm of term is uniquely identified  by numeral sequence  - path from term head.
   I. e. there exists  $subterm^*:T_v\times INT^* \to T_v$
   \begin{itemize}
     \item $subterm^*(x_i,k)=NIL$
     \item $subterm^*(c_i,k)=NIL$
     \item $$subterm^*(t(t_{1}\dots t_{n}),K)=\left\{\begin{array}{l l}
             NIL  &  length(K)=0 \\
             NIL  &  K[0] > n \\
             t_{k[0]}  &  K[0] < n \land length(K)=1\\
             subterm^*(t_{k[0]},K/K[0]) & K[0]<n \land length(K)>1 \\
            \end{array}\right.$$
   \end{itemize}
 \item As usual, free variables substitution is the set of pairs $(x_i,t_i)$, which will be denoted as  
     $substitution([x_1,t_1]\dots [x_n,t_n]) \in S$ . If it doesn't cause ambiguity it will be also denoted as 
  $s([x_1,t_1]\dots [x_n,t_n])$ or just as $s[x_i,t_i]$
 \item The substitution process $subst(t,s)$  itself is defined as follows:
  \begin{itemize} 
    \item $$
  subst(t,s([x_1,t_1]\dots [x_n,t_n]))=
             subst(t,\dots(subst(t,s[x_1,t_1])\dots,s[x_n/t_n])
    $$
    \item $subst(NIL,s)=NIL$
    \item $subst(c_i,s)=c_i$
    \item $$subst(x_i,s[x_j,t_j])=\left\{\begin{array}{l l}
         t_j & i = j \\
         x_i & i \neq j \\
        \end{array}\right.$$
    \item $subst(t(t_1\dots t_n),[x_i,t_k])=t(subst(t_1,[x_i,t_k])\dots subst(t_n,[x_i,t_k]))$
  \end{itemize} 
 \item  We define coherent unification as $bound\_unify : T_v, \times T_v \times S \to T_v \times S $ - by existing substitution and two terms we obtain unificator and adjusted substitution . 
   \begin{itemize} 
     \item $bound\_unify(NIL,x,S) = NIL $
     \item $bound\_unify(x,NIL,S) = NIL $
     \item $$bound\_unify(c_i,c_j,S) = \left\{\begin{array}{l l}
        NIL,S & c_j != c_j \\
        c_i,S & c_i = c_j \\
      \end{array}\right.$$ 
     \item $bound\_unify(x_i,t,S) = t, S\land[x_i,t]$
     \item $bound\_unify(t,x_i,S) = t, S\land[x_i,t]$
     \item $bound\_unify(x_i,x_j,S) = x_i, S\land[x_i,x_j] $
     \item $$bound\_unify(t_1(t_{11}\dots t_{1n}),t_2(t_{21}\dots t_{2m}),S) = $$
         $$=
         \left\{\begin{array}{l l}\\
           NIL,S & n!=m \\
           NIL,S & name(t_1) != name(t_2) \\
           t_1(bound\_unify(t_{11},t_{21},S)\dots bound\_unify(t_{1n},t_{2n},S)),\land S \\
         \end{array}\right.$$
   \end{itemize} 
 \item  And incoherent unification  as $free\_unify : T_v, \times T_v \to T_v \times S $ - 
  by two terms we obtain unificator and adjusted substitution , presuming that terms are 
  not bound by common variable set , i. �.
  \begin{itemize}
    \item $free\_unify(t_1,t_2)=bound\_unify(t_1,free\_fv(t_2,fv(t_1)))$
    \item $free\_fv(c_i,(x_1\dots x_n))=c_i$ 
    \item $free\_fv(x_i,(x_1\dots x_n))=\left\{\begin{array}{l l}
                x_i & i>n \\ 
                x_k, x_k\not\in fv(t_1) \land x_k\not\in fv(t_2) & \\
           \end{array}
           \right.$
  \end{itemize}
    
 \end{itemize}
 

 Rewriting rule is the three: $(X,  t_{in},t_{out})$ where:
 \begin{itemize}
  \item $X = {x_i}$ is the free variables set.
  \item $t_{in} \in T_v$  input term, such as  $fv(t_{in})=X$
  \item $t_{out} \in T_v$  output term, such as $fv(t_{out})\in X$
 \end{itemize}



  Rewriting operation semantics is defined by following expression:
   $$ apply(t, t_{in}, t_{out}) = subst(t_{out},free\_unify(t_{in},t))$$

 Let's extend rewriting operations for rules set:
  $$                                    
   apply(t,\{r_1\dots r_n\})=\{ apply(t,r_1) \dots apply(t,r_n)\} 
  $$

  and for free terms set \footnote{ free terms set is term set defined to sentential 
  symbols renaming }:
  
  $$                                    
   apply(\{t_1\dots t_n\}, R)=\cup apply(t_i,R)
  $$

 Evidently, the rule can be represented as term:
    $rule(vars,t_{in},t_{out})$ where:
   \begin{itemize}
     \item $vars$ - variables set , term of  $set(x_1,\dots x_n)$ form
     \item $t_{in}$ and  $t_{out}$ are input and output terms accordingly.
   \end{itemize}


  We will say that term set  $x$ can be rewritten into term set $y$, if there exists a chain    
   $z_1\dots z_n$ such as 
  $$z_1 = x \land z_n=y \land \forall i \in \{1\dots n\} z_{n+1}\in apply(z_n,R)$$


  At last , will say that term set $T$ converges relatively to set of rules $R$ 
  if there exists  fixed point of equation:

  \begin{itemize}
   \item $apply^*(T,R) = X:  X = apply(X,R)$
  \end{itemize}                                     

  The set of rules $R$ is called converging relatively to finite set of terms $T$, if  $apply^*(T,R)$ is finite.

  Rule set  has the noetherian property, if 
$$\forall t\in T; r_1,r_2 \in R apply(t,r_1)!=apply(t,r_2) \exists R^* \in R :
    apply(apply(t,r_1),R^*)=apply(apply(t,r_2),R^*) \land apply(t_1,R^*) is finite $$ 

  Thus we come to wide-known rewriting rules theory ,
with some differences from canonical statements:
  \begin{itemize}
    \item we operate sets not just single terms  - it makes statements more general    
     (for instance  $apply^*(X,R)$ exists always ) but with that disadvantage that the main definition is not constructive 
    \item reflexive definitions of this calculus are directly embedded into the calculus.
    \item Operations of ordered set are directly included into term calculus, and are not emulated
by the set of normalizing rules as usually.
  \end{itemize}

  Those differences don't add anything essentially new to mathematical properties of 
  algebra except that it's more convenient to work with it, from technical point of view .

\subsection{ Ordered rulesets }

 We would extend semantics of ruleset evaluation on non-confluent rule systems, by applying order of rule evaluations in critical pairs.
I. e. rules $p$, $s$ forms critical pair, if exis such term $t$, than:
 $apply(p,t)\neq apply(s,t)$.  Newman's lemna say, that if system is locally-confluent (i.e. for each criticat pair exists common reduct for $apply(p,t)$ and $apply(s,t)$ than system is confluent.  But many practically interested systens better described in terms of locally-incoherent rules with fixed order of evaluation.

 So, let's introduce partial ordering of rewriting rules by 'concretization' of input patterns.  I.e. we will say that $t1 {\le}_{c} t2$ when $t1$ is 'more concrete' than $t2$.  And what is more concrete ? - this means that $t1$ can be represented as partial case of $t2$, i.e. exists substitution of free variables $s$: $subst(t2,s) \equiv t1$. Equivalent definition: $\forall t: free\_unify(t,t1) \Rightarrow free\_unify(t,t2)$.

  Now, let $p=(in_{p} \to out_{p})$ and $s=(in_{s} \to out_{s})$ are two rules, which form critical pair. we see that $p \le_{c} s$ if $in_{p} {\le}_{c} in_{s}$, and let's fix the order as evaluation as: more concrete term evaluated first, i. e.  
$$
 apply(\{p,s\},t)=\left\{
  \begin{array}{l l}
    apply(p,t) & p \le_{c} s \\
    apply(s,t) & s \le_{c} p \\
    \iota\{apply(p,t),apply(s,t)\} & p \sim_{c} s\\
  \end{array}
 \right.
$$
 where $\iota$ is operator of non-determenistics choice, defined by strategy.

It's easy to show, that for each ruleset ${r_{1}\dots r_{n}}$ where $\forall i,j : i\neq j \Rightarrow \not {r_i} \sim_{c} r_j$ can be build equivalent locally-confluent ruleset.

\subsection{Actions}

 Next question, that should be considered is the term calculation representing. 
 Usual approach  (set of rules, that convert input term set to output) 
 doesn't suit us, becouse in real world we need something more:
 \begin{itemize}
   \item In  applied problems, unlike mathematical vacuum, there is some environment, 
   where we can obtain necessary information as required, as well as report to. 
   \item Often we don't need conversion from input to output in software packages, but we need
  description of  exchange with environment sequence.
   \item The conversion rules set can be dynamic.
 \end{itemize}

 Among the samples of such applications there are set of rules for determination of 
 the following actions in a business process, where set depends on process variables, 
 or the code generation process, that depends on certain architecture, 
 or  the output rules, presented by facts of the deductive database.                                                                

Consequently, we need to define operations semantics  by including openly environment 
 interaction and rules set modification dependent on state.


Terminal mechanism is the four:  
  $<S, E, \phi_s,\phi_e>$                                                             
 where:
 \begin{itemize}
   \item $S=<S_t,S_r>$ is state, the pair of 
   \begin{itemize}
     \item $S_t$ - term set.
     \item $S_r$ - active rewriting rules set .
   \end{itemize}
     Note, that the division is quite relative, for rewriting  rules are presented by terms of special kind.
    \item $E$  is environment representing in the system.        
    \item $\phi_s:S\times X \times E \to S\times Y$ is the system transformation function.
    \item $\phi_e: E \times Y -> E$ is the environment reaction function.
 \end{itemize}                                     

 The terminal mechanism behavior is defined by the following transformation in discrete time unit:

 $$<S,E,X> -> <\phi_s(X,E,S)|_{S},\phi_e(\phi_s(X,E,S)|_{Y})$$

(here  $x|_{Y}$ denotes projection  $x$ onto coordinate $Y$ in general sense  of  set theory).

The following diagram can be more clear.

\begin{verbatim}
          phi_s
 <S,E,X> ----------> <S',Y'>   phi_e
    |                       ----------> E'
    ---------------------E
\end{verbatim}
 
                     
\begin{itemize}
 \item $\phi_s$ is operation of applying  $S_r$ to $S_f$ according to some strategy, 
       influencing   environment  $Y$ as by-effect  (from the programming point of 
       view it will be output operation).
     The system is normal if $\phi_s$ defines fixed  point of application $S_r$ to $S_f$
 \item $\phi_e$ is the interpretation of  $env(y)$ by environment. 
\end{itemize}
              
\subsection{ Terminal System With Interactions }

What changes we should make in the term rewriting system to reflect operation 
semantics requirements  - we should just add some syntax  of environment interaction: 

Input/output pair ($X:x\to y$) is replaced by the four  ($X:(x,e_{in})\to (y,e_{out})$,
where
 \begin{itemize}
  \item $x$, $y$ are input and output terms as before.
  \item $e_{in}$ is environment state request.
  \item $e_{out}$ is environment influence operation.
 \end{itemize}
  
We will denote the four as  $x [ e_{in} ] ->  y [ e_{out} ]$.
  Expression $x[ e_{in} ]$ for term $t$ can be interpreted as comparison with 
sample $x$ while executing  $subst(e_{in},snd(free\_unify(t,x)))$ 
(let $snd(free\_unify(t,x)) = s$), 
$y[e_{out}]$ - as substitution of unification into $y$ and $subst(e_out,s)$ execution.
   


\section{ TermWare Language }


Lexical Units:
\begin{itemize}
 \item Constants are numbers and strings with usual semantics 
 \item Variables - \verb|$x|
 \item Identifiers. -
 \item Terms constructors: $f(x_0\dots x_n)$
 \item Set-Patterns are terms of  $set\_pattern(x,y)$ form, symbols - $\{ x: y \}$, unifying
with set, that consist of  unificator $�$ and of set $y$ 
\end{itemize}
              
Syntax:                                              
\begin{itemize}
 \item Just term algebra.
 \item Operators and shortenings.       
 \begin{itemize}
   \item $x \to y$ is shortening for $rule(x,y)$
   \item $x [c] \to y$ is shortening for $if_rule(x,c,y)$
   \item $x [c] \to y [r]$ is shortening for $if_rule(x,c,action(y,z))$
   \item $x.y$ is shortening for $apply(x,y)$
   \item $[x,y,\dots z]$ is shortening  for $cons(x,cons(y,\dots,z)\dots)$
   \item $\{x,y,\dots z\}$ is shortening for $set(x,y,\dots z)$
   \item Arithmetic expressions with standard priority rules (if more precisely - corresponding to C language syntax).
 \end{itemize}
\end{itemize}

 See also 
 \htmladdnormallink{BNF definitions}{\CmBase/docs/TermWareBNF.html}


\begin{itemize}
 \item Predefined systems:
  \begin{itemize} 
   \item Gen contains "natural" (from common sense point of view) rules for general  
    algebraic operations.
   \begin{itemize}
     \item plus
     \item minus
     \item apply(system-name, term);
   \end{itemize}
   \item Sys contains Java environment interaction primitives.
   \begin{itemize}              
     \item $javaFacts(name,className);$
     \item $setFact(name,t_1,...t_n);$
     \item $checkFact(domain,X);$
     %\item askFact(domain,x);
     \item $javaStrategy(name, className);$
     \item $system(name, facts, ruleset, strategy);$
     \begin{itemize}
       \item name is string or atom, which means system name
       \item facts is the name of database facts
       \item ruleset is term, that denotes rules set .
             it should be of  $ruleset(t_1,\dots,t_n)$ form, where 
             $t_i$ is 
             \begin{enumerate}
               \item or rule  $cond\_rule(x,y,z,v)$, $rule(x,y)$ is the shortening for $cond\_rule(x,NIL,y,NIL)$
               \item or term of  $import(x)$ form, meaning "import all the rules from system  with name x"
               \item or term of $importTransformed(x,y)$ form, meaning "import all the rules from system with name x, with transformation, defined by Y"
               \item or term of  $javaRule(className)$ form.
             \end{enumerate}
      \item strategy is the strategy name.
     \end{itemize}    
     \item $javaParser(javaClass);$
     \item $loadFile(fname);$
     \item $loadFile(parser,fname);$
     \item $setProperty(name,term);$
     \item $getProperty(name);$
     \item $forgetSystem(name);$
   \end{itemize}
  \end{itemize}
  \item Default is Gen and Sys combination.  
\end{itemize}

Full description of build-in systems see in
 \htmladdnormallink{TermWareAPI reference}{\TAPIBase/index.html}


Samples of systems, defined by  TermWare Language:
                    
\begin{verbatim}

System(BooleanAlgebra,default,
ruleset(
 $x & ($x => z) -> $z ,
 not($x & $y) -> not($x) | not($y) ,
 not($x | $y) -> not($x) & not($y) ,
 $x & ($y | $z) -> ($x & $y) | ($x & $z) ,
 not(not($x)) -> $x , 
), 
FirstTop);

System(BooleanLogic,general,
 ruleset( import(BooleanAlgebra),
          true => $x -> $x,
          false => $x -> not($x),
          true | $x -> true,
          false | $x -> $x,
          true & $x -> $x,
          false & $x -> false,
          not(true) -> false,
          not(false) -> true
        ), FirstTop);

 
\end{verbatim}   

Here, as you can see all rules are shortened (without operations and data base).

The full system model with operations and fact base demands facts semantics 
and operations description  (or rather their programming in Java)  and will 
be of enormous volume for this article.

Therefore below we'll cite terminal system "of real life"
\footnote{ somewhat simplified }, 
 and facts semantics and operations we'll describe informally :

\begin{verbatim}
System(BugFixing,DevelopmentProcess,
 ruleset(

 received($bug_id) -> check_confirmation($bug_id) 
                                 // human_task(check_bug($bug_id)),

 check_confirmation($bug_id) [|confirmed($tester,$bug_id)|]
                          -> known($bug_id)
                                 // human_task(fix_bug($bug_id),
                                               write_regression($bug_id)
                                              ),

 check_confirmation($bug_id) [|not_confirmed($tester,$bug_id)|]
                          -> true // send_not_confirmed($bug_id,$tester),


 known($bug_id) [|
                  fixed($developer1, $bug_id) && 
                  added_regression($developer2,$bug_id)
                 |]
                          -> true // send_closed($bug_id,$developer1)
 ),
 FirstTop)
\end{verbatim}    
 
This example describes TermWare application in business-processes organization
system. As you've guessed the example is software error message processing. 

 Here fact base (environment) can answer the questions:
  \begin{itemize}
    \item confirmed(x,y) - $x$ person confirmed $y$ error message
    \item not\_confirmed(x,y) -  $x$ denied $y$ message
    \item fixed(x,y) -  $x$ repaired software $y$ error
    \item added\_regression(x,y) - $x$ added tests for $y$ into regression tests set.
  \end{itemize}

 And can interpret messages:
   \begin{itemize}
     \item $human\_task(x)$ - ask somebody to perform $x$
     \item $send\_not\_confirmed(x)$ - report about denied error $x$
     \item $send\_closed(x)$ - report about repaired error $x$
   \end{itemize}

 Notice, that environment data are transmitted to the system by 
 sentential expressions in conditions.  More detailed this primer is described
 in \cite{ISTA2003-TermWare}.

 

\subsection{ Hierarchical Name System  }    

From terminal systems programming point of view it is convenient to use Hierarchical
Name Mechanism resembling to idl modules, C++ namespaces or Java packages.

In TermWare such mechanism is represented by \verb|domain| term. 
Syntax: $domain(name,def_1,\dots def_n)$, where  $def_i$  is systems or 
subdomains definition terms.

Example:
\begin{verbatim}
domain(algebra,
  System(Semigroup),
  domain(LiGroups,
    System ..
  )
)
\end{verbatim}

Also automatic system definitions loading from files is implemented:  
while interpreting embedded conversions in any context, if full system 
name (for instance:\verb|x.y.z|, i.e. system with \verb|z| name, which is 
located in \verb|x/y| domain) is found, then the file is automatically 
loaded from according location of file system relative to 
\verb|termware.path| - Java environment property (in case of addressing 
to \verb|x.y.z| system an attempt to read \verb|<termware.path>/x/y/z.def|
file will be performed). 


\subsection{ Embedded  Parsers Of Other Languages }  

TermWare  syntax, as it is, is not always convenient for data input:
sometimes we'd like to use more natural problem-oriented language for the purpose.

Toward this aim TermWare provides extension possibility:
users can embed syntax parsers of their own languages by  implementing verb|IParser| 
and \verb|IParserFactory|  interfaces in their Java classes and by registering 
the classes names in general dictionary.
                                   
\section{ Java API }

\subsection{ Terms Operating }

\subsubsection{Term}

Main TermWare entity is \verb|Term| interface.

 \htmladdnormallink{(API Description)}{\APIBase/Term.html}



As you can see, ITerm is interface where the following elements are defined: 
 \begin{itemize}
   \item Standard term operations, such as arity or name.
   \item Logical operations, such as  unification and equivalence.
   Notice, that we have two unifications and two equivalences:
  free and bound. Their semantics is described in Formal Model, but let's  repeat it 
  in few words - \verb|free| versions don't  discern free variables with different indexes. 
   \item Subsidiary operations. 
   \item Atomic terms detectors and operations of conversion to elementary 
            data types:
           \verb|Int|,
            \verb|String| and  etc. 
 \end{itemize}

 Unlike many computer algebra systems, term system is not build bottom-up, from 
 atomic data types, but top-down, i.e. any class, that extends abstract class \verb|Term|,
 can be declared as term.

Of course, there exists predetermined term types set (described in Formal Model).
According inheritance diagram looks as follows:
\begin{verbatim}
 Term
  |
  *---AbstractPrimitiveTerm
  |     |
  |     *--AtomTerm
  |     |
  |     *--BooleanTerm
  |     |
  |     *--DoubleTerm
  |     |
  |     *--IntTerm
  |     |
  |     *--NILTerm
  |     |
  |     *--StringTerm
  |
  *---AbstractComplexTerm
  |   |
  |   *--SetTerm
  | 
  *---XTerm
 
\end{verbatim}

But from programmer's point of view it's not necessary to use these  classes:  
it would be better for this purpose to use \verb|TermFactory| interface.

\subsubsection{TermFactory}

TermFactory  is a class, accessible from TermWare singleton,
 that provides methods to create terms. 
 \htmladdnormallink{(API Description)}{\APIBase/TermFactory.html}

 Usage examples:
\begin{verbatim}
 Term term=TermWare.getInstance().getTermFactory().createAtom("qqq");
\end{verbatim}
 �������


\begin{verbatim}
 TermFactory termFactory=TermWare.getInstance().getTermFactory();
 Term term=termFactory.createComplexTerm3("f_3",
                                  termFactory.createInt(3),
                                  termFactory.createComplexTerm2("f",
                                            termFactory.createAtom("a1"),
                                            termFactory.createAtom("a2")
                                                                 ),
                                  termFactory.createX(1)
                                                     );
\end{verbatim}
will create term $f_3(3,f(a1,a2),x_1)$ with one free variable $x_1$.
 

\subsubsection{ITermTransformer}

 It is just term converter:
 \htmladdnormallink{(API Description)}{\APIBase/ITermTransformer.html}

 Main converter functioning is concealed in \verb|transform| method, which
transform input term to output in some system and with transformation context
(which give us access to current substitution and so on).
  
Let's consider how term converters are bound with rewriting rules:
we can create according converter just by rewriting rule. 


\subsubsection{ITermRewritingStrategy}

And at last rewriting strategy:
 \htmladdnormallink{(API Description)}{\APIBase/ITermRewritingStrategy.html}
It's rather difficult object, which includes rules (term converters) set and 
the rules application algorithm. 


End user usually doesn't work directly with rewriting strategys, but application 
programmers can create their own rewriting strategy by inheriting their 
class  from \verb|AbstractTermRewritingStrategy|.
 \htmladdnormallink{(API Description)}{\APIBase/ITermRewritingStrategy.html}

This minding of strategy come from APS \cite{APS1}, another approach can be found
in Stratego \cite{Stratego}.

\subsubsection{IFacts}
                        
 IFacts 
 \htmladdnormallink{(API Description)}{\APIBase/IFacts.html}
 is terminal system knowledgebase pattern, that figures environment (or oracle)
 of our logic.

 What we can do with fact:
 \begin{itemize}
   \item - check.
   \item - set.
   \item - ask something uncertain (ask).
 \end{itemize}
 
 Naturally, knowledge base semantics is utterly defined by data domain,
 implementation should be created separately each for a problem and may include
 requests to relation database or to input/output operators. 


 Rule $x[condition]->y[action]$ is applyed to term $t$ as follow:
 \begin{enumerate}
  \item $t$ is matched with pattern $x$.
  \item On success of previous step, we call $check$ method of facts db with
    argument \verb|condition|.
  \item If \verb|set| return false, we belive that rule is not applicable.
  \item otherwise, we substitute output pattern $y$ as result 
  \item and call method \verb|set| of db facts with $action$ as argument.
 \end{enumerate}                           


 TermWare provide implementation of Facts DB with name \verb|default| in class
\verb|DefaultClass|  \htmladdnormallink{(API Description)}{\APIBase/DefaultFacts.html}

 In this class \verb|check| and \verb|set| methods implemented as follow:
 Logical and ariphmetical expressions are evaluated, if during evaluation we
 meet functional symbol, with name and arity wich matched name and arity of
 some method of own class, than this method is called. Argument and result
 are converted to Java types, if needed, using Java reflection API.

 I. e. for calling methods of you facts database from rules just derive
 you facts database from class \verb|DefaultFacts|.

                                             
\subsubsection{IEnv}

Except 'logical' environment, defined in the logic, we have also 'program' or 
'physical' environment, that defines: 
 \begin{itemize}
   \item where request user input from. 
   \item where output calculations results to.
   \item where output error messages and debug information to.
 \end{itemize}

 The environment is encapsulated in \verb|IEnv| interface
 \htmladdnormallink{(API Description)}{\APIBase/IEnv.html}

\subsubsection{TermSystem}

Now we can define Terminal  System, so: 
Terminal System is rewriting rules with strategies and fact data base and 
program environment interface totality.
 \htmladdnormallink{(API Description)}{\APIBase/TermSystem.html}

 Knowing all this we can define the term conversion operations: \verb|reduce|.
 
Also we can define methods of rules adding to system, of term reduction and so on. 
These methods have comparatively simple interface (programmers, who use TermWare
as library should start API studying exactly from these methods).  

Typical code of terminal system creating looks as follows:

\begin{verbatim}
   IEnv env=TermWare.getInstance().createEnv(
             "ua.kiev.gradsoft.TermWare.envs.SystemLogEnv", envTerm
                                  );

   ITermRewritingStrategy strategy=TermWare.getInstance().createJavaStrategy(
             "ua.kiev.gradsoft.TermWare.strategies.TopDownStrategy"
                                                                  );
   IFacts facts=TermWare.getInstance().createFacts(
             "ua.kiev.gradsoft.TermWare.facts.AttributeFacts", env
                                            );
   ITermSystem system=new TermSystem(strategy,facts,env);
\end{verbatim}


\subsection{ Subsidiary Classes }

\subsubsection{TermwareInstance}

 TermWareInstance
 \htmladdnormallink{(API Description)}{\APIBase/TermWareInstance.html}
 is , figuratively, a shank, that contains hierarchical domain system
 (see \verb|Domain| class, 
 \htmladdnormallink{(API Description)}{\APIBase/Domain.html}, \verb|getDomain()|
method), user dictionaries  of  language parsers  names  and  printers  names.

 Usially we have one instance in JVM, but it is possible to create and use few
instances when we need few unrelated theories in one JVM (for example inside application server).


\subsubsection{ IParser, IParserFactory }

You can define user language parsers embedding into TermWare system   
with the  help of these interfaces. 

 General usage pattern looks as follows:
 \begin{itemize}
   \item Application programmer defines language parser according to 
   \verb|IParser| interface
 \htmladdnormallink{(API Description)}{\APIBase/IParser.html}
     (IParserFactory).
   \item Before using the parser programmer registers his parser factory in 
   \verb|TermWare| for language \verb|X| by using 
\begin{verbatim}
  TermWareInstance.addParserFactory(String language, IParserFactory factory)
\end{verbatim}
method. 
After that verb|loadFile(fname,X)| call will use \verb|X| language parser.   
 \end{itemize}


\subsubsection{ IPrinter, IPrinterFactory }

These interfaces define embedding of term own output. The actions should be the same 
as in previous part, except instead of input stream we use output stream.
API Description:
 \begin{itemize}
  \item \htmladdnormallink{IPrinter}{\APIBase/IPrinter.html}
  \item \htmladdnormallink{IPrinterFactory}{\APIBase/IPrinter.html}
 \end{itemize}


%\section{ Supplements }

\section{ Changes }

 \begin{enumerate}
  \item 02.01.2008 - fixed few references to obsolete API.
  \item 28.06.2007 - added description of 'more concrete first' policy.
  \item 28.03.2007 - reviewed and repackaged for version 2.0.0.
  \item 07.03.2004 - cosmetics.
  \item 07.02.2004 - changed description of default facts database.
  \item 12.01.2004 - first external edition.
  \item 22.01.2003 - first internal edition.
  \item 8.10.2002 - created.
 \end{enumerate}

\bibliography{Bib} 

\end{document}


